\documentclass[10pt,a4paper]{article}

\begin{document}

\small

\begin{flushleft}
  188 \, CHAPTER THIRTEEN
\end{flushleft}

background and/or contact information for the interested researcher to build on the work in their own research.In our exploration of the Net, we discovered a site designed to assist researchers in publishing their results.The Literati Club site provides a host of hints,guidelines,and resources in addition to discussion lists and email notifications for prospective publishers.As an example of the type of resources available at the site,Robert Brown suggests an "action learning" approach to creating dissemination materials,especially peer-reviewed articles(Brown,1995).He suggests that aspriring authors organize a "learning set" ideally composed of five experienced researchers.These researchers read the proposed dissemination articles(s) and meet face-to-face to review the author's answers to the following questions :


\begin{flushleft}
  \qquad 1.Who are the intended readers ? (list three to five of them by name)

  \qquad 2.What did you do ? (limit fifty words)

  \qquad 3.Why did you do it ? (limit fifty words)

  \qquad 4.What happened ? (limit fifty words)

  \qquad 5.What do the results mean in theory ? (limite fifty words)

  \qquad 6.What do the results mean in practice ? (limit fifty words)

  \qquad 7.What is the key benefit for your readers ? (limit twenty-five words)

  \qquad 8.What remains unresolved ? (limit fifty words)

\end{flushleft}


Regardless of the actual use of a collaborative "learning set," the e-researcher should be able to answer all of these questions clearly and concisely.The answers become the framework on which the dissemination article is built.Brown notes that beginning authors tend to focus on the second and fourth questions and tend to de-emphasize those components that are of interest to most potential readers---the third,sixth,and seventh questions.

Creating clear and meaningful results is to some degree independent of the medium used to disseminate these results.However,as Marshall McLuhan reminded us,the "medium is the message," and thus we focus on the new media provided by the Net for the dissemination of results .

\begin{flushleft}
\normalsize
\textbf{ADVANTAGES OF DISSEMINATING YOUR RESULTS}
  
\textbf{VIA THE NET}
\end{flushleft}

The Net provides additional venues for disseminationg results in addition to the preNet era's focus on paper and print production of articles and press release.The Net offers obvious cost savings for dissemination in that printing and postage costs are eliminated.Production costs associated with design and formatting are,however,equal or even greater for Net-distributed material than those associated with paper-based dissemination.In addition, the capacity to add multimedia tends to set an expectation for the addition of this type of graphic,animated,or audiovisual enhancement.Multi-media producation costs can be very high and the budget conscious researcher will plan carefully and manage such expenditures judiciously.

\end{document}